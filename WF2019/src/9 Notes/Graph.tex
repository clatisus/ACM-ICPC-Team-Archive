\subsubsection{无源无汇有容量下界网络的可行流}

\begin{enumerate}
\item 建立附加源 $s$ 和汇 $t$。
\item 添加弧 $t\to s$ 并设容量为无穷大。
\item 将原图中上界为 $c$,下界为 $b$ 的弧 $u\to v$ 拆成三条弧:$s\to v$ 和 $u \to t$,容量都为 $b$,以及 $u\to v$,流量为 $c-b$。
\item 对附加弧进行合并:对于原图中每个点 $i$ 记录 $d[i]=\sum\limits_{e:s\to i} cap(e) - \sum\limits_{e:i\to t} cap(e)$。如果 $d[i]>0$,则连一条容量为 $d[i]$ 的弧 $s\to i$;如果 $d[i]<0$,则连一条容量为 $-d[i]$ 的弧 $i\to t$。
\item 求改造后的网络的s-t最大流,当且仅当所有的附加弧满载时原网络有可行流。
\end{enumerate}

\subsubsection{有源汇有容量下界网络的s-t可行流}

\begin{enumerate}
\item 建立超级源 $ss$ 和 $tt$。
\item 添加弧 $t\to s$ 并设容量为无穷大。
\item 将原图中上界为 $c$,下界为 $b$ 的弧 $u\to v$ 拆成三条弧:$ss\to v$ 和 $u \to tt$,容量都为 $b$,以及 $u\to v$,流量为 $c-b$。
\item 附加弧进行合并:对于原图中每个点 $i$ 记录 $d[i]=\sum\limits_{e:ss\to i} cap(e) - \sum\limits_{e:i\to tt} cap(e)$。如果 $d[i]>0$,则连一条容量为 $d[i]$ 的弧 $ss\to i$,如果 $d[i]<0$,则连一条容量为 $-d[i]$ 的弧 $i\to tt$。
\item 求改造后的网络的ss-tt最大流,当且仅当所有的附加弧满载时原网络有s-t可行流。
\end{enumerate}

\subsubsection{霍尔婚配定理}

左部为 $X$,右部为 $Y$ 的二分图 $G=(X+Y,E)$ 中(不妨假设$|X|\le|Y|$),存在一个匹配覆盖 $X$ 的充分必要条件是:$\displaystyle\forall S(S\subseteq X \to\left|S\right|\le\left|\bigcup\limits_{x\in S}T(x)\right|)$,其中 $T(x)=\{y|y\in Y\land\left[(x,y)\in E\right]\}$。

\subsubsection{广义霍尔定理}

二分图 $A+B$ 中存在一个匹配能覆盖顶点集合 $X+Y$ 的充要条件是存在一个匹配能覆盖 $X$,且存在一个匹配能覆盖 $Y$($X\in A\land Y\in B$)。

\subsubsection{对偶图最小生成树}

对偶图最小生成树,等于平面图所有边减去平面图最大生成树。

\subsubsection{Euler Characteristic}

$$
\chi = V-E+F 
$$

其中,V为点数,E为边数,F为面数,对于平面图即为划分成的平面数(包含外平面),

$ \chi $为对应的欧拉示性数,对于平面图有 $ \chi = C + 1 $,C为连通块个数。

\subsubsection{Maxflow on Planar Graph}

连接s和t,显然不影响图的平面性,转对偶图,令原图中s和t连接产生的新平面在对偶图中对应的节点为s',外平面对应的顶点为t',删除s'和t'之间直接相连的边。
此时s'到t'的一条最短路就对应了原图上s到t的一个最大流。

\subsubsection{Prüfer sequence}

根据树构造:我们通过不断地删除顶点编过号的树上的叶子节点直到还剩下 2 个点为止的方法来构造这棵树的 Prüfer sequence。特别的,考虑一个顶点
编过号的树$T$, 点集为$ { 1, 2, 3, \ldots , n} $。在第i步中,删除树中编号值最小的叶子节点,设置Prüfer sequence的第i个元素为与这个叶子节
点相连的点的编号。

还原:设 $a_i$ 是一个 Prüfer sequence。这棵树将有 $n + 2$ 个节点, 编号从 $1$ 到 $n + 2$ , 对于每个节点,计它在 Prüfer sequence 中出现的次数 +1 为其度数。然后, 对于 $a$ 中的每个数 $a_i$ , 找编号最小的度数值为 $1$ 节点 $j$ ,加入边 $(j,a_i)$ , 然后将 $j$ 和 $a_i$ 的度数值减少 $1$。最后剩下两个点的度数值为 $1$,连起来即可。

结论:完全图$K_n$的生成树, 顶点的度数必须为$d_1, d_2, \ldots, d_n$,这样的生成树棵数为:
$$ \frac{(n-2)!}{[(d_1 - 1)! (d_2 - 1)! (d_3 - 1)! \ldots (d_n - 1)!]} $$

一个顶点编号过的树, 实际上是编号的完全图的一棵生成树。 通过修改枚举Prüfer sequence的方法,可以用类似的方法计算完全二分图的
生成树棵数。 如果G是完全二分图, 一边有$n_1$个点,另一边有$n_2$个点,则其生成树棵数为 $n_1 ^ {n_2 - 1} * n_2 ^ {n_1 - 1}$。

\subsubsection{Best Theorem}

边编号,则有向图中本质不同的欧拉回路数量为:
$$
ec(G) = t_w(G) \prod_{v \in V}{(\deg(v) - 1)!}
$$
其中,$t_w(G)$ 表示以 $w$ 为根的树形图个数,deg 是入度,这里 $w$ 可以任选。通过 Matrix-Tree theorem 可知,有向图中以 $w$ 为根的树形图个数,等于该图对应的基尔霍夫矩阵对于 $w$ 所在行列的余子式。其中基尔霍夫矩阵等于入度矩阵减去反图的邻接矩阵。

\subsubsection{2-Sat}

判断变量 $x_i$ 是否可以取 $1$,如果 $x_i$ 和 $\lnot x_i$ 不在一个连通块,那么 $x_i$ 可以取任意值;否则在一个连通块,那么 $x_i$ 在 $\lnot x_i$ 拓扑序后面的时候可以取 $1$。模板中只需要判断 \text{bel[i] < bel[i + n]} 即可。

\subsubsection{Matrix Tree}

无向图关联矩阵 $B$,行是点,列是边。若 $e_k=(v_i,v_j)$ 那么 $B_{ik},B_{jk}$ 中一个为 $1$ 一个为 $-1$。它的 Kirchhoff 矩阵为 $BB^T$,也即度数矩阵 $-$ 邻接矩阵。

任意图 $G$ 的 Kirchhoff 矩阵 $C$ 的行列式为 $0$(行/列向量和为 $0$ 向量)

不连通图 $G$ 的 Kirchhoff 矩阵 $C$ 的任一个 $n-1$ 阶主子式行列式为 $0$

若 $G$ 是一棵树,任一个 $n-1$ 阶主子式行列式为 $1$

根据 Cauchy–Binet formula,把 $C_r=B_rB_r^T$ 展开,得到:

$$
\begin{aligned}
|C_r|=|B_rB_r^T|&=\sum_{S\subset \{1,2,\dots,m\},|S|=n-1}|(B_r)_S||(B_r)_S^T|\\
&=\sum_{S\subset \{1,2,\dots,m\},|S|=n-1}|(B_r)_S|^2
\end{aligned}
$$

若 $S$ 中 $n-1$ 条边形成了树,那么贡献 $1$;否则贡献 $0$。推广:

自环删掉,重边 $(i,j)$ 有 $m$ 条边,则 $C_{ij}=C_{ji}=-m$,度数同样要计算在内。

有向图,入度矩阵 $-$ 邻接矩阵,行向量之和为 $0$ 向量。

边带权,令 $C_{ij}=C_{ji}=-w_{ij},C_{ii}=-\sum_{i\neq j}C_{ij}$。则 $n-1$ 阶主子式的值为:$\sum_{T\text{ form a tree }}\left(\prod_{e\in T}w_e\right)$

求恰有 $k$ 条特殊边的生成树个数,把 $k$ 条边边权设为 $x$,那么 $n-1$ 阶主子式的值是一个多项式,对应阶的系数即为答案。可以用插值而不是多项式高消求解。

\subsubsection{杂项}

\textbf{Block Forest}: 为每个点双新建虚点,与点双中每个点连边,形成一个森林,叶子都是非割点,内部都是割点和虚点,连通性等价。

\textbf{Bipartite}: 二分图最小路径覆盖 = 最大独立集 = 总节点数 - 最大匹配数,最小点覆盖 = 最大匹配数。任意图中,最大独立集 + 最小点覆盖集 = $|V|$ ,最大团 = 补图的最大独立集。

\textbf{一般图匹配随机做法}: 记录每个点匹配点,随机多次重排边表后用匈牙利算法增广,多次都没找到增广路认为无法匹配。卡掉它的图必须有许多奇环加上大量无用边,例如奇数点连成团,偶数点只和其附近奇数点连边。

\textbf{三元环}: 对 $(degree, index)$ 排降序,排名靠前的对排名靠后的连边,先枚举 $u\rightarrow v$ 标记 $v$,再枚举 $u\rightarrow v\rightarrow w$,其中 $w$ 被标记。

\textbf{四元环}: 排序同上,枚举 $u$,以及按照排名降序枚举邻点 $v$,再枚举排名比 $u,v$ 都大的 $w$,先统计 $w$ 的答案,然后标记它。

\textbf{弦图}: 任意超过 $3$ 个点的环都存在弦,弦即环上不相邻两点之间的边。完美消除序列满足 $v_i$ 和 $v_{i+1},\dots,v_n$ 构成的诱导子图里,把和 $v_i$ 相连的点拿出来,会成为一个团。求法:初始化 $d_i = 0$,每次选取 $d$ 值最大的点放在序列前部,再枚举该点的出边,把他们的 $d$ 值加 $1$,如此反复。性质: 从后往前染色可得图的色数、最大团大小,从前往后贪心取可得最大独立集、最小团覆盖。弦图判定: 枚举 $v_i$,看它往后连的点里最靠前的点是不是和之后的点都有边。完美图: 所有诱导子图满足色数等于最大团大小的图。伴完美图: 所有诱导子图满足最大独立集等于最小团覆盖的图。区间图总是弦图,按右端点排序可得完美消除序列。

\textbf{差分约束系统}: $x_j − x_i\le b_k$ 即 $d[v] − d[u] \le w[u, v]$。以每个变量 $x_i$ 为结点连边,增加源点 $s$ 与所有定点连边权为 $0$ 的边,以 $s$ 为源点运行 Bellman-ford 算法(或SPFA算法),最终 $\{d[i]\}$ 即为一组可行解。最大值求最长路,最小值求最短路。求最短路时如果路径上有负权值,则要判负环。求最长路时如果路径上有正权值,则要判正环。不等号的方向要统一。$d[i]$ 值看情况初始化。

\textbf{Dancing Links}: 精确覆盖时,选列覆盖,去掉相关行,从列链表枚举行选法。重复覆盖时,选列覆盖,从列链表枚举行选法,去掉相关列。估计为贪心选。