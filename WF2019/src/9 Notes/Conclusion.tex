\subsubsection{数学分析}

\begin{enumerate}

\item

$$
\frac{\pi}{2} = \lim_{n \to +\infty}\frac{[\frac{(2n)!!}{(2n-1)!!}]^{2}}{2n+1}
$$

\item

$$
n!\sim\sqrt{2\pi n}(\frac{n}{e})^{n}
$$

\item $n$维球的体积公式:$\displaystyle{\frac{\pi^{\frac{n}{2}}}{\Gamma(\frac{n}{2}+1)}r^{n}}$

$n$维球的表面积公式:$\displaystyle{\frac{2\pi^{\frac{n}{2}}}{\Gamma(\frac{n}{2})}r^{n-1}=V_{n}'(r)}$

其中伽马函数 $\Gamma(z)=\int_{0}^{\infty}x^{z-1}e^{-x}\mathbb{d}x$ ,满足

 - $\Gamma(1)=1,\Gamma(z+1)=z\Gamma(z)(z>0)$
 - $\Gamma(1-z)\Gamma(z)=\frac{\pi}{\sin(\pi z)}(z\notin\mathbb{Z})$
 - $\Gamma(z)\Gamma(z+\frac{1}{2})=2^{1-2z}\sqrt{\pi}\Gamma(2z)$
 - $\Gamma(\frac{1}{2})=\sqrt{\pi}$

\end{enumerate}

\subsubsection{组合数学}

\begin{enumerate}

\item 设 $p(n)$ 为将 $n$ 写成若干个正整数和的方案数,若 $i$ 为自然数,称 $\frac{3i^{2}-i}{2}$ 和 $\frac{3i^{2}+i}{2}$ 为广义五边形数,并定义 $f(\frac{3i^{2}-i}{2}) = f(\frac{3i^{2}+i}{2}) = i$,则 $p(n) = \sum_{u,1 \le u \le n}(-1)^{f(u) - 1}p(n-u)$,其中 $u$ 为广义五边形数。$\displaystyle{\phi(x)=\prod_{i=1}^{\infty}(1-x^{i})}$ 称为欧拉函数,它是划分数生成函数的逆。$\phi(x) = \sum_{u}(-1)^{f(u)}x^{u}$,其中 $u$ 为广义五边形数。

\item $\begin{bmatrix}n\\k\end{bmatrix}$ 表示 $n$ 个元素的置换中能被分解为 $k$ 个循环的置换个数,并定义 $s(n,k)=(-1)^{n-k}\begin{bmatrix}n\\k\end{bmatrix}$。 $\begin{bmatrix}n+1\\k\end{bmatrix}=n\begin{bmatrix}n\\k\end{bmatrix}+\begin{bmatrix}n\\k-1\end{bmatrix}(k>0)$,$\begin{bmatrix}0\\0\end{bmatrix}=1$,$\begin{bmatrix}n\\0\end{bmatrix}=\begin{bmatrix}0\\n\end{bmatrix}=0(n>0)$。
该数列满足:

- $(x)^{(n)}=\sum_{k=0}^{n}\begin{bmatrix}n\\k\end{bmatrix}x^{k}$
- $(x)_{n}=\sum_{k=0}^{n}s(n,k)x^{k}$

\item $\begin{Bmatrix}n\\k\end{Bmatrix}$ 表示有 $n$ 个元素的集合划分为 $k$ 个集合的方案数。$\begin{Bmatrix}n+1\\k\end{Bmatrix}=k\begin{Bmatrix}n\\k\end{Bmatrix}+\begin{Bmatrix}n\\k-1\end{Bmatrix}(k>0)$,$\begin{Bmatrix}0\\0\end{Bmatrix}=1$,$\begin{Bmatrix}n\\0\end{Bmatrix}=\begin{Bmatrix}0\\n\end{Bmatrix}=0(n>0)$。
该数列满足:

- $\begin{Bmatrix}n\\k\end{Bmatrix}=\frac{1}{k!}\sum_{i=0}^{k}(-1)^{i}{k\choose i}(k-i)^{n}$

\item 设 $G$ 是一个有限群,作用于集合 $X$ 上,对 $\forall g\in G$,$X^{g}$ 表示 $X$ 中在 $g$ 作用下的不动元素的集合,$|X/G|$ 表示 $X$ 在 $G$ 作用下的轨道数,则有 $|X/G|=\frac{1}{|G|}\sum_{g\in G}|X^{g}|$。

\item 设 $\bar{G}$ 是 $n$ 个对象的一个置换群,用 $m$ 种颜色涂染这 $n$ 个对象,则不同染色的方案数为 $l=\frac{1}{|\bar{G}|}(m^{c(\bar{a_{1}})}+m^{c(\bar{a_{2}})}+\cdots+m^{c(\bar{a_{g}})})$ ,其中 $c(\bar{a_{i}})$ 表示置换 $\bar{a_{i}}$ 的循环节数。

\item $\sum{n\choose2k}{k\choose m}=\frac{n}{n-m}{n-m\choose n-2m}2^{n-2m-1}$

\end{enumerate}

\subsubsection{数论}

\begin{enumerate}

\item 对 $\forall a, b, n \in N^{+}, b \ge \phi(n)$,有 $a ^ {b} \equiv a ^ {b \% \phi(n) + \phi(n)}\pmod{n}$,注意 $b \ge \phi(n)$ 是必要条件,以及式子中取模后指数必须加上 $\phi(n)$,否则结果可能会出错。

\item

$$
\prod_{i=1,\gcd(i,m)=1}^{m}i\equiv\begin{cases}
-1&(m=1,2,4,p^{l},2p^{l},\text{ p is odd prime})\\
1&(\text{otherwise})
\end{cases}\pmod{m}
$$

\end{enumerate}

\subsubsection{计算几何}

\begin{enumerate}

\item 设球冠的高为 $h$,半径为 $R$,则表面积为 $2\pi Rh$。

球面三角形的面积为 $(A+B+C-\pi)R^{2}$ 。

球面正弦定理:$\frac{\sin A}{a} = \frac{\sin B}{b} = \frac{\sin C}{c}$

球面余弦定理:$\cos a=\cos b\cos c+\sin b\sin c\cos A$, $\cos A=-\cos B\cos C+\sin B\sin C\cos a$

\item 设有一格点多边形,其面积 $S=a+\frac{b}{2}-1$,其中 $a$ 表示多边形内部的整点数,$b$ 表示多边形边界上的整点数。

\item 将 $y=kx+b$ 反演到 $(-k,b)$,将 $(k,b)$ 反演到 $y=-kx+b$。这样,两点共线反演为两条直线交于一点,该点是原直线反演出的点。

\item 平面图转对偶图时,先将每条边拆成两条有向边。从每条未访问的边开始,例如 $u->v$,那么应该寻找 $v$ 的出边中顺时针方向离 $v->u$ 最近的边,它就是该区域 $u->v$ 逆时针方向的下一条边。无界区域需要用有向面积为负来判断。点定位时,找出该点上方最近的边,如果它上方恰有多条边相交,那么找出斜率最小的边,该边 $x$ 坐标减少方向的有向边所在的区域,即是该点所在的区域。使用扫描线加速。

\end{enumerate}

\subsubsection{其它}

\begin{enumerate}

\item $x_{n+1}=x_{n}-\frac{f(x_{n})}{f'(x_{n})}$

\end{enumerate}

\subsubsection{无源无汇有容量下界网络的可行流}

\begin{enumerate}

\item 建立附加源 $s$ 和汇 $t$。

\item 添加弧 $t\to s$ 并设容量为无穷大。

\item 将原图中上界为 $c$,下界为 $b$ 的弧 $u\to v$ 拆成三条弧:$s\to v$ 和 $u \to t$,容量都为 $b$,以及 $u\to v$,流量为 $c-b$。

\item 对附加弧进行合并:对于原图中每个点 $i$ 记录 $d[i]=\sum\limits_{e:s\to i} cap(e) - \sum\limits_{e:i\to t} cap(e)$。如果 $d[i]>0$,则连一条容量为 $d[i]$ 的弧 $s\to i$;如果 $d[i]<0$,则连一条容量为 $-d[i]$ 的弧 $i\to t$。

\item 求改造后的网络的s-t最大流,当且仅当所有的附加弧满载时原网络有可行流。

\end{enumerate}

\subsubsection{有源汇有容量下界网络的s-t可行流}

\begin{enumerate}

\item 建立超级源 $ss$ 和 $tt$。

\item 添加弧 $t\to s$ 并设容量为无穷大。

\item 将原图中上界为 $c$,下界为 $b$ 的弧 $u\to v$ 拆成三条弧:$ss\to v$ 和 $u \to tt$,容量都为 $b$,以及 $u\to v$,流量为 $c-b$。

\item 附加弧进行合并:对于原图中每个点 $i$ 记录 $d[i]=\sum\limits_{e:ss\to i} cap(e) - \sum\limits_{e:i\to tt} cap(e)$。如果 $d[i]>0$,则连一条容量为 $d[i]$ 的弧 $ss\to i$,如果 $d[i]<0$,则连一条容量为 $-d[i]$ 的弧 $i\to tt$。

\item 求改造后的网络的ss-tt最大流,当且仅当所有的附加弧满载时原网络有s-t可行流。

\end{enumerate}

\subsubsection{霍尔婚配定理}

左部为 $X$,右部为 $Y$ 的二分图 $G=(X+Y,E)$ 中(不妨假设$|X|\le|Y|$),存在一个匹配覆盖 $X$ 的充分必要条件是:$\displaystyle\forall S(S\subseteq X \to\left|S\right|\le\left|\bigcup\limits_{x\in S}T(x)\right|)$,其中 $T(x)=\{y|y\in Y\land\left[(x,y)\in E\right]\}$。

\subsubsection{广义霍尔定理}

二分图 $A+B$ 中存在一个匹配能覆盖顶点集合 $X+Y$ 的充要条件是存在一个匹配能覆盖 $X$,且存在一个匹配能覆盖 $Y$($X\in A\land Y\in B$)。

\subsubsection{Euler Characteristic}
$$ \chi = V-E+F $$
其中,V为点数,E为边数,F为面数,对于平面图即为划分成的平面数(包含外平面),
$ \chi $为对应的欧拉示性数,对于平面图有 $ \chi = C + 1 $,C为连通块个数。

\subsubsection{Dual Graph}
将原图中所有平面区域作为点,每条边若与两个面相邻则在这两个面之间连一条边,只与一个面相邻连个自环,若有权值(容量)保留。

\subsubsection{Maxflow on Planar Graph}
连接s和t,显然不影响图的平面性,转对偶图,令原图中s和t连接产生的新平面在对偶图中对应的节点为s',外平面对应的顶点为t',删除s'和t'之间直接相连的边。
此时s'到t'的一条最短路就对应了原图上s到t的一个最大流。

\subsubsection{根据树构造}
我们通过不断地删除顶点编过号的树上的叶子节点直到还剩下2个点为止的方法来构造这棵树的Prüfer sequence。特别的,考虑一个顶点
编过号的树$T$, 点集为$ { 1, 2, 3, \ldots , n} $。在第i步中,删除树中编号值最小的叶子节点,设置Prüfer sequence的第i个元素为与这个叶子节
点相连的点的编号。

\subsubsection{还原}
设 $a_i$ 是一个Prüfer sequence。
这棵树将有 $n + 2$ 个节点, 编号从 $1$ 到 $n + 2$ , 对于每个节点,计它在Prüfer sequence中出现的次数+1为其度数。
然后, 对于$a$中的每个数$a_i$, 找编号最小的度数值为1节点$j$, 加入边$(j,a_i)$ , 然后将j和a_i的度数值减少1。最后剩下两个点的度数值为1,连起来即可。

\subsubsection{一些结论}
完全图$K_n$的生成树, 顶点的度数必须为$d_1, d_2, \ldots, d_n$,这样的生成树棵数为:
$$ \frac{(n-2)!}{[(d_1 - 1)! (d_2 - 1)! (d_3 - 1)! \ldots (d_n - 1)!]} $$

一个顶点编号过的树, 实际上是编号的完全图的一棵生成树。 通过修改枚举Prüfer sequence的方法,可以用类似的方法计算完全二分图的
生成树棵数。 如果G是完全二分图, 一边有$n_1$个点,另一边有$n_2$个点,则其生成树棵数为 $n_1 ^ {n_2 - 1} * n_2 ^ {n_1 - 1}$。

\subsubsection{Matrix Tree}
对于$n$个点的无向图的生成树计数,
令矩阵 $D$ 为图 $G$ 的度数矩阵,即 $ D = diag(deg_1,deg_2, \ldots , deg_n) $, $A$ 为 $G$ 的邻接矩阵表示,
则 $ D - A $ 的任意一个 $n-1$ 阶主子式的行列式的值即为答案。

\subsubsection{Maximum Clique}
最大团,复杂度上限大概是 $ O(3^{\frac{n}{3}}) $ 左右。G是邻接矩阵,$dp_i$表示由$i$到$n-1$的子图构成的最大团点数,剪枝用。adj中存放的是与i邻接且标号比i大的顶点。

顺便可以保证方案的字典序最小。另外对adj进行压位有非常良好的效果。




